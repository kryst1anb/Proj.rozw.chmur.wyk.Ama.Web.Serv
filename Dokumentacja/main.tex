\documentclass[12pt]{report}
\usepackage[english]{babel}
\usepackage{polski}
\usepackage{url}
\usepackage[utf8x]{inputenc}
\usepackage{amsmath}
\usepackage{graphicx}
\graphicspath{{images/}}
\usepackage{parskip}
\usepackage{xcolor}
\usepackage{fancyhdr}
\usepackage{vmargin}
\usepackage{listings}
\usepackage{anyfontsize}
\usepackage{color}
\definecolor{bluekeywords}{rgb}{0.13,0.13,1}
\definecolor{greencomments}{rgb}{0,0.5,0}


\newcommand\tab[1][1cm]{\hspace*{#1}}
\setmarginsrb{1.5 cm}{1.5 cm}{1.5 cm}{1.5 cm}{1 cm}{1.0 cm}{1 cm}{1.0 cm}

\addto\captionsenglish{\renewcommand*\contentsname{Spis treści}}
\addto\captionsenglish{\renewcommand\bibname{Bibliografia}}

\title{RecognizeMe - rozpoznawanie twarzy Amazon Rekognition}								\author{Aleksander Boronowski}

\makeatletter
\let\thetitle\@title
\let\theauthor\@author
\let\thedate\@date
\makeatother

\pagestyle{fancy}
\fancyhf{}

\lhead{\thetitle}
\cfoot{\thepage}
\usepackage{tikz}
\usetikzlibrary{matrix,chains,positioning,decorations.pathreplacing,arrows,calc}
 
\tikzset{
block/.style={
  draw,
  rectangle,
  text width=3em,
  text centered,
  minimum height=8mm,    
  node distance=2.3em
  },
line/.style={draw}
}

\lstset{
basicstyle=\small,
keywordstyle=\color{black}\bfseries\underbar,
identifierstyle=,
stringstyle=\ttfamily, 
showstringspaces=false,
basicstyle=\scriptsize}

\begin{document}

\begin{titlepage}
	\centering
    \vspace*{0.3 cm}
	 
\begin{center}    \textbf{\LARGE RecognizeMe\\ \vspace*{0.3 cm} rozpoznawanie twarzy Amazon Rekognition}\\[0.8 cm]	\end{center}
   \includegraphics[scale = 0.40]{logo.png}\\[0.6 cm]
	\begin{minipage}{0.4\textwidth}
		\begin{flushleft} \large
			\end{flushleft}
			\end{minipage}~
			\begin{minipage}{0.6\textwidth}
			\begin{flushright} \large
			\emph{Autorzy:} \\
			Marta Lewandowska,\\
			Krystian Barczak,\\
			Aleksander Boronowski,\\
			Wydział Matematyki Stosowanej\\
			Kierunek Informatyka\\
			VI semestr - grupa KUTAR3
		\end{flushright} \vspace{1.2cm}
          Gliwice, 2019/2020 
	\end{minipage}\\[2 cm]
\end{titlepage}

\tableofcontents
\pagebreak
\renewcommand{\thesection}{\arabic{section}}

\hyphenation{użytkownika wyświetli przeglądarki aplikacji przycisk przejrzystego wymagającego efektywną}

\section{Opis programu}
\hspace{0.5cm}Webowa aplikacja pozwalająca na rozpoznanie twarzy wraz ze szczegółami z podanego przez użytkownika zdjęcia. Aplikacja wykorzystuje usługę AWS Amazon Rekognition.

\vspace*{0.3cm}
\hspace{0.5cm}Program został wykonany w celu projektu zaliczeniowego z przedmiotu Projektowanie rozwiązań chmurowych z wykorzystaniem Amazon Web Services.
\vspace*{0.5cm}
\section{Cel i geneza projektu}
\hspace{0.5cm}Celem projektu było stworzenie prostej aplikacji webowej, w której można znajdować i analizować osoby na zdjęciu podanym przez użytkownika wykorzystując usługę AWS Rekognition.
\vspace*{0.5cm}
\section{Zespół projektowy}
\begin{enumerate}
    \item \textbf{Marta Lewandowska}\\
    Rola: Grafik \\
    Zadania: Frontend, tworzenie filmów, tworzenie prezentacji  \\
    \item \textbf{Krystian Barczak}\\
    Rola: Programista\\
    Zadania: Backend, tworzenie filmów, tworzenie prezentacji\\
    \item \textbf{Aleksander Boronowski }\\
    Rola: Tester\\
    Zadania: Tester, bugmenadżer, tworzenie dokumentacji, tworzenie filmów, tworzenie prezentacji\\
\end{enumerate}
\newpage
\section{Instrukcja obsługi}
\begin{enumerate}
    \item \textbf{Upload własnego zdjęcia}\\\\
W celu dodania zdjęcia, które użytkownik chce sprawdzić programem, należy kliknąć przycisk \textbf{"Upload photo"}, a następnie wybrać plik obrazu. Jeśli plik nie jest obrazem, wyświetli się błąd z nazwą podanego pliku.
\vspace*{0.5 cm}
\begin{center}

\includegraphics[scale = 0.375]{main1.jpg}\\[0.4 cm]

\includegraphics[scale = 0.515]{error.jpg}\\[0.4 cm]
\end{center}
\newpage
    \item \textbf{Odczytywanie danych}\\\\
Po poprawnym załadowaniu zdjęcia pod wyświetlonym obrazem użytkownika wyświetli się tabela z uzupełnionymi danymi. Pokazuje ona ilość wykrytych osób na zdjęciu oraz procent \\w jakim dany atrybut jest prawdopodobny dla zaznaczonej osoby. Jeśli na zdjęciu znajduję się dużo osób, tabele można przesuwać za pomocą scroll bara aby zobaczyć wszystkie dane.
\\
\vspace*{0.5cm}
\begin{center}
\includegraphics[scale = 0.5]{table.jpg}\\[0.2 cm]
\end{center}
\newpage
    \item \textbf{Wyświetlanie danych}\\\\
Użytkownik może wybrać, które dane mają wyśietlić się w tabelii. Wystarczy kliknąć przycisk opcji po prawej stronie oraz zaznaczyć lub odznaczyć interesujące go funkcje.
\\
\vspace*{0.5cm}
\begin{center}
\includegraphics[scale = 0.5]{settings.png}
\end{center}
\end{enumerate}
\newpage
 \section{Specyfikacja techniczna}
 \vspace*{1.0 cm}
 \textbf{Podział na pliki:} \\ \\
 \begin{center}
     \includegraphics[scale = 1.0]{pliki.jpg}\\[0.6 cm]
 \end{center}
 \vspace*{1,5cm}
 \textbf{Kompilacja projektu: }\\
 
\hspace{0.5cm} Do stworzenia projektu wykorzystany został program Microsoft Visual Studio Code oraz przeglądarki Google Chrome i Opera. Aplikacja napisana została w języku HTML oraz CSS. Funkcjonalność aplikacji natomiast napisana została w języku JavaScript, PHP i Python oraz framework jQuery.
     \newpage
     \section{Szczegóły techniczne}
\begin{enumerate}
\item \textbf{Funkcja odpowiedzialna za pobieranie danych JSON}\\ \\
\begin{lstlisting}
function getJSON() {
  xmlhttp = new XMLHttpRequest();
  xmlhttp.open("GET", fileNameJSON, true);

  xmlhttp.onreadystatechange = function () {
    if (xmlhttp.readyState === XMLHttpRequest.DONE) {
      var status = xmlhttp.status;
      if (status === 0 || (status >= 200 && status < 400)) {
        document.getElementsByClassName("functionBox")[0].style.display = "none";
        document.getElementsByClassName("loader")[0].style.display = "none";
        clearInterval(refreshIntervalId);

        myArr = JSON.parse(this.responseText);
        lenghtOfPeople = myArr.FaceDetails.length;
        sendJSON(myArr);
      } else {
        var refreshIntervalId = setInterval(() => {
          if (status === 0 || (status >= 200 && status < 400)) {
            document.getElementsByClassName("functionBox")[0].style.display = "none";
            document.getElementsByClassName("loader")[0].style.display = "none";
            clearInterval(refreshIntervalId);
          } else {
            status = 0;
            getJSON();
          }
        }, 1500);
      }
    }
  };
  xmlhttp.send();
}
\end{lstlisting}

\item \textbf{Funkcja odpowiedzialna za wysyłanie danych JSON do serwera PHP}\\ \\
\begin{lstlisting}
function sendJSON(myArr) {
  xmlhttp = new XMLHttpRequest();
  xmlhttp.open("POST", "server.php", true);
  xmlhttp.onreadystatechange = function () {
    if (xmlhttp.readyState === XMLHttpRequest.DONE) {
      var status = xmlhttp.status;

      if (status === 0 || (status >= 200 && status < 400)) {
        document.getElementsByClassName("inner")[0].innerHTML = this.responseText;
        document.getElementById("photo").width = img.width;
        document.getElementById("photo").height = img.height;
        drawAll();
      } else {
        console.log("Oh no! There has been an error with the request!");
      }
    }
  };
  xmlhttp.send(
    JSON.stringify({
      send: true,
      fileNameToPy: file.name,
      length: myArr.FaceDetails.length,
      json: myArr,
    })
  );
}
\end{lstlisting}

\item \textbf{Funkcja odpowiedzialna za rysowanie Canvas oraz zaznaczanie na nim twarzy}\\ \\
\begin{lstlisting}
function drawAll() {
  clearVariables();
  for (var i = 0; i < lenghtOfPeople; ++i) {
    imgHeight = document.getElementById("photo").height;
    imgWidth = document.getElementById("photo").width;

    var canvas = document.getElementById("photo");
    var ctx = canvas.getContext("2d");
    ctx.beginPath();
    ctx.lineWidth = "3";

    if (imgWidth > 500 && imgHeight > 300) {
      ctx.lineWidth = "6";
      ctx.font = "Bold 48px Comic Sans MS";
      ctx.fillText(i + 1, imgWidth * ratioX, imgHeight * ratioY - 10);
    }

    var getColor = randomColor();
    document.getElementById("colName_" + i).style.color = getColor;
    ctx.strokeStyle = getColor;
    ctx.fillStyle = getColor;

    var json = myArr["FaceDetails"][i]["BoundingBox"];

    var ratioX = 0;
    ratioX = json.Left;

    var ratioY = 0;
    ratioY = json.Top;
    var ratioWidth = 0;
    ratioWidth = json.Width;
    var ratioHeight = 0;
    ratioHeight = json.Height;

    ctx.rect(imgWidth * ratioX, imgHeight * ratioY, imgWidth * ratioWidth, imgHeight * ratioHeight);
    ctx.stroke();
    ctx.closePath();
  }
  document.getElementsByClassName("canvas-photo")[0].style.display = "block";
  document.getElementsByClassName("main-tab")[0].style.display = "block";
}
\end{lstlisting}

\item \textbf{Funkcja odpowiedzialna za ustawienie obrazka}\\ \\

\begin{lstlisting}
function setBackgroundAndName(file) {
  xmlhttp = new XMLHttpRequest();
  xmlhttp.open("post", "upload.php", true);
  xmlhttp.onreadystatechange = function () {
    document.getElementById("photo").style.backgroundImage = "url(upload/" + file.name + ")";
  };
  var data = new FormData();
  data.append("file", file);
  fileWithoutExt = file.name.replace(/\.[^/.]+$/, "");
  fileNameJSON = "upload/" + fileWithoutExt + ".json";
  xmlhttp.send(data);
}
\end{lstlisting}
\newpage
\item \textbf{Funkcje odpowiedzialne za generowanie koloru}\\ \\

\begin{lstlisting}
function random(min, max) {
  var num = Math.floor(Math.random() * (max - min)) + min;
  return num;
}

function randomColor() {
  return "rgb(" + random(0, 255) + ", " + random(0, 255) + ", " + random(0, 255) + ")";
}
\end{lstlisting}

\item \textbf{Funkcja odpowiedzialna za nasłuchiwanie zmian elementu Input}\\ \\

\begin{lstlisting}
document.getElementById("file").addEventListener(
  "change",
  function () {
    file = this.files[0];
    if (file.type.split("/")[0] === "image") {
      document.getElementsByClassName("functionBox")[0].style.display = "block";
      document.getElementsByClassName("loader")[0].style.display = "block";
      document.getElementsByClassName("errorFormatter")[0].style.display = "none";
      setBackgroundAndName(file);
      fileName = file.name;
      getJSON();
      var _URL = window.URL || window.webkitURL;
      if ((file = this.files[0])) {
        img = new Image();
        img.src = _URL.createObjectURL(file);
      }
    } else {
      document.getElementsByClassName("loader")[0].style.display = "none";
      document.getElementsByClassName("functionBox")[0].style.display = "block";
      document.getElementsByClassName("errorFormatter")[0].style.display = "block";
      document.getElementsByClassName("errorFormatter")[0].style.color = "red";
      document.getElementsByClassName("errorFormatter")[0].innerHTML = 
      "Not a valid file: " + file.name;
    }
  },
  false
);
\end{lstlisting}
\newpage
\item \textbf{Funkcja odpowiedzialna za wywołanie pliku Python}\\ \\

\begin{lstlisting}
<?php

$target_path = "upload/" ;
if(!file_exists($target_path)) {
    mkdir($target_path, 0755, true);
}

$filename = basename($_FILES['file']['name']);
$tmp_name = $_FILES['file']['tmp_name'];
$target_path = $target_path . basename( $_FILES['file']['name']);

if(move_uploaded_file($tmp_name, $target_path)) {
    $procedure = ".aws/start.py ". $filename;
    exec($procedure);
} else{
    echo "There was an error uploading the file, please try again!";
}
    exit;
?>
\end{lstlisting}
\newpage
\item \textbf{Funkcja odpowiedzialna za dynamiczne generowanie tabeli z przetworzonymi danymi}\\ \\

\begin{lstlisting}
<?php
$f = fopen("uneditable","a");
flock($f,LOCK_EX);

$rawdata = file_get_contents("php://input");
$dataJSON = json_decode($rawdata,true);
$ok = true;


if($dataJSON == null) {
    $result = array('status' => false, 'code' => 1, 'value' => 'Bad format');
    $ok = false;
}

$dataJSON['json']['length'] = $dataJSON['length'];

if($dataJSON['send'] == true){
	foreach($dataJSON['json']['FaceDetails'] as $chunk) {
		unset($chunk["Landmarks"],$chunk["Pose"],$chunk["BoundingBox"]);
	}

	foreach ($dataJSON['json'] as $key => $value) {
		if($key === "FaceDetails"){
			$intcols = count($value);
		}
	}

	$array = ['AgeRange','Smile','Eyeglasses','Sunglasses','Gender','Beard',
	'Mustache','EyesOpen','MouthOpen','Confidence'];
	$arrayEmotions = ['Happy','Calm','Sad','Surprised','Disguisted','Fear','Angry','Confused'];

	echo "<table id='table'>";
	echo "<td class='headcol'>Name of Attribute</td>";
	for($j=0;$j<$intcols;$j++) {
		$id = $j +1;
		echo "<td class='long' id='colName_".$j."'>Person ".$id."</td>";
	}

	for ($i = 0;$i<10;$i++) {

		echo "<tr id='".$i."'>";
		echo "<td class='headcol'>".$array[$i]."</td>";

		foreach($dataJSON['json']['FaceDetails'] as $chunk) {
			if ($chunk[$array[$i]] == $chunk['AgeRange']){
				echo "<td class='long'>".$chunk['AgeRange']
				['Low']."-".$chunk['AgeRange']
				['High']."</td>";
			}
			else if($chunk[$array[$i]] == $chunk['Gender']){
				if($chunk['Gender']['Value'] == "Male"){
					echo "<td class='long'>Male in ".round($chunk['Gender']
					['Confidence'],2)."%</td>";
				}
				else {
					echo "<td class='long'>Female in ".round($chunk['Gender']
					['Confidence'],2)."%</td>";
				}
			}
			else if($chunk[$array[$i]] == $chunk['Confidence']){
				echo "<td class='long'>" .round($chunk['Confidence'], 2)."%</td>";
			}
			else if($chunk[$array[$i]] != $chunk['AgeRange']){
				if($chunk[$array[$i]]['Value'] == 1){
					echo "<td class='long'>Has in ".round($chunk[$array[$i]]
					['Confidence'],2)."%</td>";
				}
				else {
					echo "<td class='long'>Has not in ".round($chunk[$array[$i]]
					['Confidence'],2)."%</td>";
				}
			}
		}
		echo "</tr>";
	}
	
	for ($j = 0;$j<8;$j++) {
		$rowNumber = $j+10;
		echo "<tr id='".$rowNumber."' style='display:none;'>";
		echo "<td class='headcol'>".$arrayEmotions[$j]."</td>";
	
		foreach($dataJSON['json']['FaceDetails'] as $chunk) {
			echo "<td class='long'>".round($chunk['Emotions'][$j]
			['Confidence'],2)."%</td>";
		}
	
		echo "</tr>";
	}
	echo "</table>";
}


	flock($f, LOCK_UN); 
	fclose($f);
	unlink('uneditable');
?>
\end{lstlisting}

\item \textbf{Skrypt Python odpowiedzialny za komunikację z serwerem AWS}\\ \\

\begin{lstlisting}
import boto3
import json
import sys

reko = boto3.client('rekognition')

pic =  str(sys.argv[1]) 

pic_w_ext = pic[:-4]
pic_json = "upload/"+pic_w_ext+".json" 

in_file = open("upload/"+pic, "rb") 
pic_binary = in_file.read()
in_file.close()

response_binary = reko.detect_faces(
	Image={
        'Bytes': pic_binary
    },
    Attributes=[
        'ALL',
    ]
)

with open(pic_json, 'w') as f:
    json.dump(response_binary, f)

\end{lstlisting}
\end{enumerate}
\newpage
\section{Oczekiwane rezultaty projektu}
\hspace{0.5cm}Oczekiwanym rezultatem projektu było stworzenie kompletnego, w pełni działającego projektu zarówno w wersji webowej jak i mobilnej. Dodatkowo oczekiwane było stworzenie łatwego i przejrzystego interfejsu graficznego, a także prostota w obsłudze aplikacji przez użytkownika. Jednym z najważniejszych rezultatów było poprawne skonfigurowanie usługi AWS Rekognition w taki sposób aby uzyskiwać dane odpowiednie dla założeń naszej aplikacji.
\vspace*{0.5cm}
\section{Realizacja projektu}
\begin{enumerate}
    \item Przegląd usług Amazon;
\item Przedyskutowanie wyboru usługi;
\item Wybór Amazon Rekognition;
\item Przegląd dokumentacji usługi;
\item Implementacja połączenia między aplikacją a usługą;
\item Implementacja serwera zarządzającego odpowiedzią;
\item Implementacja wyglądu aplikacji;
\item Testowanie rozwiązań;
\item Naprawianie błędów wynikających z testowania;
\item Stworzenie filmu prezentującego działanie;
\item Stworzenie dokumentacji projektowej.
\end{enumerate}
\newpage
\section{Wnioski}
\begin{enumerate}
    \item \textbf{Spostrzeżenia}\\
    Początkowo projekt sprawiał wrażenie mocno rozbudowanego, skomplikowanego oraz\\ wymagającego dużej implementacji. Ostatecznie ta myśl okazała się błędna co skutkowało dość miłą i efektywną pracą całego zespołu.\\
    \item \textbf{Osiągnięcia}\\
    Stworzenie w pełni działającej aplikacji według postanowionych warunków i celów.\\
    \item \textbf{Potencjał rozwoju}\\
Aplikacja nie wykorzystuje całego zasobu usługi ze względu na jej wielkość. W przyszłości możliwe jest rozbudowanie aplikacji o kolejne elementy usługi Amazon Rekognition jakim jest np. rozpoznawanie celebrytów ze zdjęć.
\end{enumerate}
\begin{thebibliography}{111}
\bibitem{awsrekguide}
    {\it https://docs.aws.amazon.com/rekognition/latest/dg/what$-$is.html}
    
    \bibitem{awsrek}
    {\it https://aws.amazon.com/rekognition/}
    \bibitem{rekhelpcode}
    {\it https://docs.aws.amazon.com/AWSJavaScriptSDK/latest/AWS/Rekognition.html}
    \bibitem{stack}
    {\it https://stackoverflow.com}
\end{thebibliography}
\end{document}

